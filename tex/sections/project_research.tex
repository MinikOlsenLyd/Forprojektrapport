\chapter{Project Research}
Undersøgelse af tilsvarende projekter og relevant litteratur.
This section investigates similar projects and products. Market analysis was conducted in the fall of 2018 to create an overview of the market. 70+ key figures from the Danish and international music business - musicians, recording producers, touring personnel, etc. - were interviewed for an investigation of their consumption patterns, gear run downs, their requests for useful features and the pitfalls of their gear. \newline

The most experienced users are the so-called \textit{backline technicians} (or backliners) that travel/tour with a touring artist or band to set up, tear down and maintain the artists' instruments and gear. In addition to setting up and tearing down the gear, the backline technician often operates electronics that run parts of the show such as backing tracks on some chosen device, MIDI changes, tune guitars/basses during the perfomance - In general to assist in making the artist's performance to run smoothly without unnecessay interruptions. \newline

The most experienced backliners actively explore and research new technologies and products in between tours and/or during their spare time. Some even modify hardware and software to suit their clients' needs. And some backliners - usually on tours with larger production budgets - even design customized systems. \newline

Indeed, touring personnel - whether they are backliners, sound engineers, lighting engineers, etc. - are passionate about their job and they continually explore solutions to make a performance to run even more smoothly. \newline

The common denominator in the interviews is the critical need for stability. Some of the technologies outlined later in this section have trade-offs: Some sacrifice user-friendly GUIs to ensure stability, some trade-offs are loss in audio quality to ensure stability and user-friendly GUI and vise versa. \newline

As several of the available technologies heavily rely on laptops and home-made systems during performances, the incentive to produce Showman is to remove laptops from the perfomance scenario to avoid hiccups during a show. \newline

Following subsections briefly introduce the most popular tehcnoloqies used currently by touring artists and their personnel with a list of their pros and cons. This chapter concludes with an introduction of Showman's desired features. \\

HUSK AT FINDE MATRICEN FRA PITCH!!!

\subsection{Similar Projects}
This subsection investigates similar projects from Scandinavia. \\

\subsubsection{Interactive Performance System}
Mircea Gabriel Eftemie, Medialogy-alumni from Aalborg University developed a system called '\textbf{Interactive Performance System}', a visual system for concert performances: https://videnskab.dk/kultur-samfund/heavy-metal-pa-universitetet. The project was developed using Open Source technology as his Bachelor's Project with fellow students Thomas Wisbech, Lasse Wingreen and Brenly Bernard. \newline

The purpose of the system was to display theatrical and visual elements during performance without the aid of technical personnel to enhance the performance experience for the audience. The system combined two inputs: A static and dynamic input. The static inputs were prepared beforehand. The static inputs were controlled by an input during the performance partly from a webcam that filmed the audience and the artists and partly from a microphone that recorded and transmitted the audio on stage. \newline

Mircea played guitar in now-defunct Mnemic that utilized the Interactive Performance System. The project was put to rest after Mnemic's dissolution and the project status remains unclear. \\

\subsubsection{The Dark Player}
Members of Swedish band Darkane developed a system for backing track playback device called '\textbf{The Dark Player}' after expressing dissatifaction with current products available. I am in the process of tracking down the members to investigate their system further. \\

\subsection{Similar Products}
Products similar to Showman exist in many forms: From simple iPods to complex systems, artists can pick and choose any system available in several configurations depending on financial status. However, some proves to be too unreliant, unstable or reliant on other pieces of hardware to function as intended. The subsubsections below outlines the most widely used products. This subsection conludes with Showman's intended features. \\

\subsubsection{Cymatic Audio LP-16}


\subsubsection{Ableton Live}


\subsubsection{QLab}


\subsubsection{MAINSTAGE}


\subsubsection{Logic Pro X}


\subsubsection{Open Source Products}


\textbf{BlackBeagle}


\textbf{Arduino}


\textbf{Raspberry Pi}


\textbf{Linux Ubuntu Studio}


\subsubsection{Showman}


\subsection{Literature}

